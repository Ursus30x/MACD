\documentclass[12pt, letterpaper]{article}
\usepackage[T1]{fontenc}
\usepackage{amsmath}
\usepackage{graphicx}  

\title{Metody Numeryczne - Projekt 1:\newline Wskaznik MACD}
\author{Jakub Szymczyk, 198134}
\date{Marzec 2025}
\begin{document}
\maketitle

\section{Wstep}
\par
Głównym celem projektu była implementacja wskaźnika MACD oraz analiza jego przydatności w automatycznym podejmowaniu decyzji o kupnie lub sprzedaży akcji.
\bigskip
\par Analiza wskaźnika została przeprowadzona na notowaniach indeksu giełdowego S\&P 500 oraz kursie Bitcoina.
\bigskip
\par Projekt został zrealizowany w języku Python z wykorzystaniem bibliotek pandas, numpy oraz matplotlib. Biblioteki te służyły jedynie do wczytywania danych oraz wizualizacji wyników, natomiast wszystkie obliczenia zostały zaimplementowane bezpośrednio w kodzie źródłowym projektu.
\section{Czym jest MACD}
Wskaźnik MACD (\textit{Moving Average Convergence/Divergence}) został opracowany przez Gerarda Appela w 1979 roku i stanowi jedno z najpopularniejszych narzędzi analizy technicznej. Jego głównym celem jest identyfikacja trendów rynkowych poprzez analizę zbieżności i rozbieżności dwóch średnich wykładniczych (EMA – \textit{Exponential Moving Average}).

\bigskip
Wskaźnik MACD obliczany jest jako różnica wartości krótkoterminowej i długoterminowej średniej wykładniczej, najczęściej z okresami 12 i 26 dni. Dodatkowo, na wykresie często uwzględnia się tzw. linię sygnałową, będącą 9-okresową średnią wykładniczą wartości MACD, która pomaga w interpretacji sygnałów kupna i sprzedaży.

\bigskip
Podstawowe zasady interpretacji wskaźnika MACD obejmują:
\begin{itemize}
\item \textbf{Przecięcie linii MACD i linii sygnałowej} – sygnał kupna pojawia się, gdy MACD przecina linię sygnałową od dołu, natomiast sygnał sprzedaży – gdy przecina ją od góry.
\item \textbf{Przecięcie poziomu zerowego} – gdy MACD przechodzi powyżej zera, wskazuje na rosnącą siłę trendu wzrostowego, a gdy spada poniżej zera, może sygnalizować początek trendu spadkowego.
\item \textbf{Dywergencje} – jeśli cena akcji lub indeksu osiąga nowe szczyty, ale MACD nie potwierdza tego wzrostu, może to wskazywać na osłabienie trendu i potencjalną zmianę kierunku.
\end{itemize}

\bigskip
Dzięki swojej konstrukcji wskaźnik MACD jest często wykorzystywany w strategiach automatycznego handlu oraz analizie algorytmicznej.

\section{Implementacja}

Wskaźnik MACD został zaimplementowany w następujący sposób:
\begin{equation}
    MACD = EMA_{12} - EMA_{26}
\end{equation}
\begin{equation}
    SIGNAL = EMA_9(MACD)
\end{equation}

gdzie $EMA_N$ oznacza wykładniczą średnią kroczącą (ang. \textit{Exponential Moving Average}) obliczaną według wzoru:
\begin{equation}
    EMA_N^{today} = (p \cdot \alpha) + EMA_N^{yesterday} \cdot (1 - \alpha)
\end{equation}

gdzie:
\begin{itemize}
    \item $p$ – cena zamknięcia w danym przedziale czasowym,
    \item $\alpha$ – współczynnik wygładzający, określony jako:
    \begin{equation}
        \alpha = \frac{2}{N+1}
    \end{equation}
    \item $N$ – liczba okresów.
\end{itemize}

Który został wyprowadzony z:
\begin{equation}
    EMA_N(i) = \frac{x_i + (1 - \alpha)x_{i-1} + (1 - \alpha)^2 x_{i-2} + \dots + (1 - \alpha)^i x_0}
    {1 + (1 - \alpha) + (1 - \alpha)^2 + \dots + (1 - \alpha)^i}
\end{equation}

\section{Dane do analizy}

Do analizy wykorzystujemy notowania dwóch instrumentów finansowych:

\subsection{Indeks giełdowy S\&P 500}
Analizujemy indeks S\&P 500 w okresie od końca 2015 do końca 2024. Jest to indeks, na którym bazuje wiele funduszy ETF, co umożliwia nam ogólną analizę skuteczności działania wskaźnika MACD. 

\begin{figure}[h!]
    \centering
    \includegraphics[width=\linewidth]{S&P 500 from 2015-01-02 VALUE.pdf}
    \caption{Wykres indeksu S\&P 500 od roku 2015 do 2019}
    \includegraphics[width=\linewidth]{S&P 500 from 2019-05-01 VALUE.pdf}
    \caption{Wykres indeksu S\&P 500 od roku 2019 do 2023}
    \label{fig:sp500}
\end{figure}
\vspace{10cm}  
\subsection{Kryptowaluta Bitcoin}
Analizujemy notowania Bitcoina w okresie od końca 2014 do 2022. Ten instrument pozwala na przetestowanie wskaźnika MACD w bardzo zmiennych warunkach rynkowych.

\begin{figure}[h!]
    \centering
    \includegraphics[width=\linewidth+0.4]{BTC from 2015-01-01 VALUE.pdf}
    \caption{Wykres wartości BTC od roku 2014 do 2017}
    \label{fig:btc_2014_2017}
\end{figure}

\section{Relacja MACD do danych wejsciowych}

\subsection{S\&P 500 2014-2019}
\begin{figure}[h!]
    \centering
    \includegraphics[width=\linewidth]{S&P 500 from 2015-01-02 BUYSELL.pdf}
    \includegraphics[width=\linewidth]{S&P 500 from 2015-01-02 MACD.pdf}
    \caption{Wykresy wartosci oraz wskaznika MACD dla S\&P 500 2015-2019}
    \label{fig:sp500_2014_2017}
\end{figure}

\begin{itemize}
\item Sygnały \textbf{BUY} występowały na początku wzrostowych trendów, co pozwalało na korzystne wejście na rynek. Sygnały \textbf{SELL} zazwyczaj skutecznie wskazywały na momenty osłabienia trendu wzrostowego lub rozpoczęcia spadków.
\item Przecięcie poziomu zerowego w górę potwierdzało wzrostową dynamikę rynku, a spadek poniżej tego poziomu zapowiadał tendencję spadkową.
\end{itemize}
\vspace{10cm}  
\subsection{S\&P 500 2019-2023}
\begin{figure}[h!]
    \centering
    \includegraphics[width=\linewidth]{S&P 500 from 2019-05-01 BUYSELL.pdf}
    \includegraphics[width=\linewidth]{S&P 500 from 2019-05-01 MACD.pdf}
    \caption{Wykresy wartosci oraz wskaznika MACD dla S\&P 500 2019-2023} 
    \label{fig:sp500_2019_2023}
\end{figure}

\begin{itemize}
\item W tym okresie MACD dobrze identyfikował zarówno krótkoterminowe korekty, jak i długotrwałe trendy.
\item Silne odchylenia MACD wskazywały na dynamiczne ruchy cenowe, co okazało się użyteczne w przewidywaniu okresów dużej zmienności.
\end{itemize}

\vspace{5cm}  % Zmniejszamy przestrzeń, żeby tekst był bliżej wykresu
\subsection{BTC 2015-2019}

\begin{figure}[h!]
    \centering
    \includegraphics[width=\linewidth]{BTC from 2015-01-01 BUYSELL.pdf}
    \includegraphics[width=\linewidth]{BTC from 2015-01-01 MACD.pdf}
    \caption{Wykresy wartosci oraz wskaznika MACD dla BTC 2015-2019}
    \label{fig:BTC_2014_2017}
\end{figure}


\begin{itemize}
\item Sygnały \textbf{BUY} i \textbf{SELL} były mniej przewidywalne niż w przypadku indeksu S\&P 500, co wynika z większej zmienności rynku kryptowalut.
\item W okresach dynamicznych wzrostów MACD generował liczne sygnały, co wymagało dodatkowej analizy w celu uniknięcia fałszywych alarmów.
\end{itemize}


\vspace{10cm}  % Zmniejszamy przestrzeń, żeby tekst był bliżej wykresu
\subsection{Podsumowanie}
Wskaźnik MACD wykazał się wysoką skutecznością w identyfikacji trendów na rynku S\&P 500, chociaż przy dużych zmianach na rynku powodował więcej fałszywych sygnałów, a co gorsza, często prowadził do dużych strat. Natomiast w przypadku Bitcoina, z powodu jego wysokiej zmienności (ang. volatility), możemy zauważyć, że generował więcej fałszywych sygnałów niż nawet w S\&P 500. Połączenie MACD z dodatkowymi wskaźnikami technicznymi oraz analizą fundamentalną może znacząco zwiększyć trafność prognoz inwestycyjnych.


\section{MACD na mniejszych zakresach}
\subsection{S\&P 500 – 8 transakcji}
\begin{figure}[h!]
    \centering
    \includegraphics[width=\linewidth]{S&P 500 SHORT PERIOD from 2015-05-18 BUYSELL.pdf}
    \includegraphics[width=\linewidth]{S&P 500 SHORT PERIOD from 2015-05-18 GAINS.pdf}
    \caption{Wykresy zyskow MACD S\&P 500 2015-2019}
    \label{fig:BTC_2014_2017}
\end{figure}

\begin{itemize}
    \item Analiza krótkoterminowych sygnałów MACD dla indeksu S\&P 500 wskazuje, że strategia oparta na 8 transakcjach przynosiła niewielkie, ale stabilne zyski.
    \item Wykresy pokazują, że momenty kupna i sprzedaży były dobrze dopasowane do lokalnych szczytów i dołków rynku.
    \item Zyski były relatywnie niskie, co sugeruje, że MACD w krótkim terminie może nie być wystarczająco skuteczne samodzielnie. Dodatkowo widać dużą stratę w momencie gwałtownego spadku.
\end{itemize}
    

\subsection{BTC – 8 transakcji}

\begin{figure}[h!]
    \centering
    \includegraphics[width=\linewidth]{BTC SHORT PERIOD from 2015-03-08 BUYSELL.pdf}
    \includegraphics[width=\linewidth]{BTC SHORT PERIOD from 2015-03-08 GAINS.pdf}
    \caption{Wykresy zyskow MACD  BTC 2015-2019}
    \label{fig:BTC_2014_2017}
\end{figure}

\begin{itemize}
    \item W przypadku Bitcoina MACD generował więcej fałszywych sygnałów niż dla S\&P 500.
    \item Zmienność rynku kryptowalut powodowała, że momenty kupna i sprzedaży były mniej przewidywalne.
    \item Strategia 8 transakcji nie przyniosła stabilnych zysków, a w niektórych przypadkach generowała straty.
\end{itemize}

\vspace{10cm} 
\section{Symulacja portfela z MACD}

\subsection{Symulacja S\&P 500 2014-2019}
\begin{figure}[h!]
    \centering
    \includegraphics[width=\linewidth]{S&P 500 from 2015-01-02 COMBINED.pdf}
    \caption{Symulacja z wykresami dla S\&P 2015-2019}
    \label{fig:BTC_2014_2017}
\end{figure}
Wykresy wskazują, że strategia MACD pozwoliła osiągnąć lepsze wyniki niż jednorazowe kupno i trzymanie aktywów do końcówki roku 2017,
 gdzie potem można zaobserwować stagnację zysków, co może być powiązane z większymi wahaniami wartości. 
 Wartość portfela z transakcjami opartymi na MACD rosła stopniowo, ale nieznacznie przewyższała pasywną strategię inwestycyjną.

\vspace{10cm} 
\subsection{Symulacja S\&P 500 2019-2023}
\begin{figure}[h!]
    \centering
    \includegraphics[width=\linewidth]{S&P 500 from 2019-05-01 COMBINED.pdf}
    \caption{Symulacja z wykresami dla S\&P 2019-2023}
    \label{fig:BTC_2014_2017}
\end{figure}

W przeciwieństwie do poprzedniego okresu, MACD po spadku wynikającym z pandemii COVID-19 nie mógł powrócić do wartości konkurujących lub przewyższających wartości strategii BUY and HOLD.
\vspace{10cm} 
\subsection{Symulacja BTC 2014-2019}
\begin{figure}[h!]
    \centering
    \includegraphics[width=\linewidth]{BTC from 2015-01-01 COMBINED.pdf}
    \caption{Symulacja z wykresami dlaBTC 2015-2019}
    \label{fig:BTC_2014_2017}
\end{figure}
Wykresy pokazują ogromne wahania wartości portfela, co potwierdza, że Bitcoin jest znacznie bardziej niestabilny niż S\&P 500.
Wartość portfela opartego na MACD wahała się od kilku milionów do ponad 20 milionów dolarów, co sugeruje dużą ekspozycję na ryzyko.
MACD generował wiele fałszywych sygnałów, co w krótkim terminie mogło prowadzić do nieoptymalnych decyzji inwestycyjnych i praktycznie zerowego zysku.

\subsection{Podsumowanie}
MACD sprawdzał się lepiej dla indeksu S\&P 500 niż dla Bitcoina, co nie zmienia faktu, że nadal był gorszy niż strategia BUY and HOLD.
Może to być skorelowane z wahaniami rynku – im większe wahania, tym łatwiej wprowadzić MACD w błąd.
\vspace{2cm} 
\section{Podsumowanie}
\begin{itemize}
    \item \textbf{Efektywność MACD} – Wskaźnik MACD jest skuteczny w analizie trendów dla indeksów giełdowych, takich jak S\&P 500. Natomiast nie nadaje się do samodzielnego podejmowania decyzji – może być składnikiem algorytmu lub systemu podejmującego decyzje o kupnie lub sprzedaży aktywów. W przypadku kryptowalut, takich jak Bitcoin, jego skuteczność jest ograniczona przez wysoką zmienność rynku, która generuje bardzo duże błędy dla wskaźnika.
    \item \textbf{Krótkoterminowe vs. długoterminowe podejście} – W krótkich interwałach czasowych oraz stabilnym rynku MACD mogło sprawdzić się w zdobywaniu zysku na krótkich wahaniach. W dłuższym terminie strategia MACD generowała duże straty względem strategii BUY and HOLD, ponieważ duży spadek mógł spowodować bardzo wysokie straty, które były trudne do odzyskania.
    \item \textbf{Możliwości ulepszenia strategii} – Dodatkowe wskaźniki analizy technicznej mogą poprawić skuteczność transakcji. Strategie łączące MACD z analizą fundamentalną mogą zapewnić lepszą przewidywalność trendów rynkowych. MACD na pewno może być składnikiem większego systemu podejmującego decyzje o kupnie i sprzedaży aktywów finansowych.
\end{itemize}
Wnioski sugerują, że MACD może być użyteczne jako element strategii inwestycyjnej, ale nie powinno być jedynym wskaźnikiem przy podejmowaniu decyzji inwestycyjnych.
\end{document}